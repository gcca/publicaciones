\documentclass[spanish]{article}

\usepackage{babel}
\usepackage[utf8]{inputenc}
\usepackage{amsmath, amsthm, amssymb, amsfonts}

\theoremstyle{plain}
\newtheorem{theorem}{Teorema}[section]
\newtheorem{lemma}[theorem]{Lema}
\newtheorem{proposition}[theorem]{Proposición}
\newtheorem{corollary}[theorem]{Corolario}
\theoremstyle{definition}
\newtheorem{definition}{Definición}[section]
\newtheorem{example}{Ejemplo}
\newtheorem{exercise}{Ejercicio}
\theoremstyle{remark}
\newtheorem{remark}{Nota}
\newtheorem{notation}{Notación}

%% \newtheorem{theorem}{Theorem}[section]
%% \newtheorem{lemma}[theorem]{Lemma}
%% \newtheorem{proposition}[theorem]{Proposition}
%% \newtheorem{corollary}[theorem]{Corollary}

%% \newenvironment{proof}[1][Proof]{\begin{trivlist}
%% \item[\hskip \labelsep {\bfseries #1}]}{\end{trivlist}}
%% \newenvironment{definition}[1][Definition]{\begin{trivlist}
%% \item[\hskip \labelsep {\bfseries #1}]}{\end{trivlist}}
%% \newenvironment{example}[1][Example]{\begin{trivlist}
%% \item[\hskip \labelsep {\bfseries #1}]}{\end{trivlist}}
%% \newenvironment{remark}[1][Remark]{\begin{trivlist}
%% \item[\hskip \labelsep {\bfseries #1}]}{\end{trivlist}}

%% \newcommand{\qed}{\nobreak \ifvmode \relax \else
%%       \ifdim\lastskip<1.5em \hskip-\lastskip
%%       \hskip1.5em plus0em minus0.5em \fi \nobreak
%%       \vrule height0.75em width0.5em depth0.25em\fi}

\author{cristHian Gz. (gcca)}
\date{}


\title{A1, Notas 2: Los enteros}
\begin{document}
\maketitle


\section{Los enteros}

\begin{definition}
  Un entero es una expresión de la forma $a\!-\!\!\!-b$, donde $a$ y $b$ son números naturales. Dos enteros son iguales, $a\!-\!\!\!-b = c\!-\!\!\!-d$, si y solo si $a + d = c + b$.
\end{definition}

\begin{definition}
  La suma de dos enteros, $a\!-\!\!\!-b + c\!-\!\!\!-d$, es definido por $(a\!-\!\!\!-b) + (c\!-\!\!\!-d) := (a + c)\!-\!\!\!-(b + d)$. El producto de dos enteros, $a\!-\!\!\!-b \times c\!-\!\!\!-d$, es definido por $(a\!-\!\!\!-b) \times (c\!-\!\!\!-d) := (ac + bd)\!-\!\!\!-(ad + bc)$.
\end{definition}


\begin{lemma}
  Dado de los números naturales $a ,b, c ,d, a', b'$. Si $(a\!-\!\!\!-b) = (a'\!-\!\!\!-b')$, entonces $(a\!-\!\!\!-b) + (c\!-\!\!\!-d) = (a'\!-\!\!\!-b') + (c\!-\!\!\!-d)$ y $(a\!-\!\!\!-b) \times (c\!-\!\!\!-d) = (a'\!-\!\!\!-b') \times (c\!-\!\!\!-d)$. Además, $(c\!-\!\!\!-d) + (a\!-\!\!\!-b) = (c\!-\!\!\!-d) + (a'\!-\!\!\!-b')$ y $(c\!-\!\!\!-d) \times (a\!-\!\!\!-b) = (c\!-\!\!\!-d) \times (a'\!-\!\!\!-b')$.
\end{lemma}


\end{document}
