\documentclass[spanish]{article}

\usepackage{babel}
\usepackage[utf8]{inputenc}
\usepackage{amsmath, amsthm, amssymb, amsfonts}

\theoremstyle{plain}
\newtheorem{theorem}{Teorema}[section]
\newtheorem{lemma}[theorem]{Lema}
\newtheorem{proposition}[theorem]{Proposición}
\newtheorem{corollary}[theorem]{Corolario}
\theoremstyle{definition}
\newtheorem{definition}{Definición}[section]
\newtheorem{example}{Ejemplo}
\newtheorem{exercise}{Ejercicio}
\theoremstyle{remark}
\newtheorem{remark}{Nota}
\newtheorem{notation}{Notación}

%% \newtheorem{theorem}{Theorem}[section]
%% \newtheorem{lemma}[theorem]{Lemma}
%% \newtheorem{proposition}[theorem]{Proposition}
%% \newtheorem{corollary}[theorem]{Corollary}

%% \newenvironment{proof}[1][Proof]{\begin{trivlist}
%% \item[\hskip \labelsep {\bfseries #1}]}{\end{trivlist}}
%% \newenvironment{definition}[1][Definition]{\begin{trivlist}
%% \item[\hskip \labelsep {\bfseries #1}]}{\end{trivlist}}
%% \newenvironment{example}[1][Example]{\begin{trivlist}
%% \item[\hskip \labelsep {\bfseries #1}]}{\end{trivlist}}
%% \newenvironment{remark}[1][Remark]{\begin{trivlist}
%% \item[\hskip \labelsep {\bfseries #1}]}{\end{trivlist}}

%% \newcommand{\qed}{\nobreak \ifvmode \relax \else
%%       \ifdim\lastskip<1.5em \hskip-\lastskip
%%       \hskip1.5em plus0em minus0.5em \fi \nobreak
%%       \vrule height0.75em width0.5em depth0.25em\fi}

\author{cristHian Gz. (gcca)}
\date{}


\title{A1, Notas 1: Los números naturales}
\begin{document}
\maketitle


\section{Axiomas de Peano}

Sea $0$ un número natural y $n\!+\!\!+$ el ``sucesor'' de $n$. Los Axiomas de Peano son:

\begin{enumerate}
\item $0$ es un número natural.

\item Si $n$ es un número natural, entonces $n\!+\!\!+$ también es un número natural.

\item $0$ no es el sucesor de ningún número natural; es decir, $n\!+\!\!+ \neq 0$ para todo número natural $n$.

\item Diferentes números naturales tienen diferentes sucesores; es decir, si $n, m$ son números naturales y $n \neq m$, entonces $n\!+\!\!+ \neq m\!+\!\!+$.

\item (Principio de Inducción) Dada una propiedad $P(n)$ de un número natural $n$. Suponiendo que $P(0)$ es cierto, y suponiendo que, siempre que $P(n)$ es cierto, $P(n\!+\!\!+)$ también es cierto. Entonces $P(n)$ es cierto para cada número natural $n$.
\end{enumerate}




\subsection{Uso del Principio de Inducción}

\emph{Afirmación.} Cierta propiedad $P(n)$ es cierta para cada número natural $n$.

\emph{Prueba.} Usando inducción. Primero se verificará el caso base $n = 0$, es decir, probar $P(0)$. (Insertar prueba de $P(0)$ aquí). Luego, suponiendo inductivamente que $n$ es un número natural y $P(n)$ ha sido demostrada. Probar que $P(n\!+\!\!+)$. (Insertar prueba de $P(n\!+\!\!+)$ aquí, asumiendo que $P(n)$ es cierto). Esto cierra la inducción; así $P(n)$ es cierto para todo númerto natural $n$.




\subsection{Adición}

Definición. Dado un número natural $m$. Para sumar cero, se define $0 + m = m$. Suponiendo, inductivamente, que se tiene definido como agregar $n$ a $m$. Entonces para agregar $n\!+\!\!+$  a $m$ se define $(n\!+\!\!+) + m := (n+m)\!+\!\!+$.

\begin{lemma}
  Para cualquier número natural $n$, $n + 0 = n$.
\end{lemma}

\begin{proof}
  Usando inducción. El caso base, $0 + 0 = 0$ se sigue de $0 + m = m$ para todo $m$ y $0$ es un número natural. Suponiendo inductivamente que $n + 0 = n$ se desea mostrar que $(n\!+\!\!+) + 0 = n\!+\!\!+$. Por definición de la adición, $(n\!+\!\!+) + 0 = (n + 0)\!+\!\!+$, que es igual a $n\!+\!\!+$ dado $n + 0 = n$. Lo que cierra la inducción.
\end{proof}

\begin{lemma}
  Para cualesquiera números naturales $n$ y  $m$, $n + (m\!+\!\!+) = (n + m)\!+\!\!+$.
\end{lemma}

\begin{proof}
  Por inducción sobre $n$ (con $m$ fijo). Para probar el caso base $n = 0$ se necesita mostrar que $0 + (m\!+\!\!+) = (0 + m)\!+\!\!+$. Por la definición de la adición, $0 + (m\!+\!\!+) = m\!+\!\!+$ y $0 + m = 0$; así, ambas partes son $m\!+\!\!+$. Luego, suponiendo inductivamente $n + (m\!+\!\!+) = (n + m)\!+\!\!+$, se necesita probar que $(n\!+\!\!+) + (m\!+\!\!+) = ((n\!+\!\!+) +m)\!+\!\!+$. El lado izquierdo, por definición de la adición es $(n + (m\!+\!\!+))\!+\!\!+$, que es igual a $((n + m)\!+\!\!+)\!+\!\!+$ por la hipótesis de inducción. Por definición de la adición es $(n\!+\!\!+) + m = (n + m)\!+\!\!+$, así el lado derecho es $((n + m)\!+\!\!+)\!+\!\!+$. Así ambos lados son iguales; lo que cierra la inducción.
\end{proof}

\begin{proposition}[Conmutativa]\label{prop:ad.conmut}
  Para cualesquiera números naturales $n$ y $m$, $n + m = m + n$.
\end{proposition}

\begin{proof}
  Por inducción sobre $n$ ($m$ fijo). El caso base, $n = 0$ , es $0 + m = m + 0$. Por definición de la adición, $0 + m = m$; por el Lema 1, $m + 0 = m$. Así, el caso base es cierto. Luego, suponiendo inductivamente, $n + m = m + n$, se mostrará que $(n\!+\!\!+) + m = m + (n\!+\!\!+)$. Por la definición de la adición, $(n\!+\!\!+) + m = (n + m)\!+\!\!+$. Por el Lema 2, $m + (n\!+\!\!+) = (m + n)\!+\!\!+$, que es igual a $(n + m)\!+\!\!+$ por la hipótesis de induccion $n + m = m + n$. Así, $(n\!+\!\!+) + m = m + (n\!+\!\!+)$ y se cierra la inducción.
\end{proof}

\begin{proposition}[Asociativa]\label{prop:ad.asoc}
  Para cualesquiera números naturales $a, b, c$, se tiene $(a + b) + c = a + (b+ c)$.
\end{proposition}

\begin{proof}
  ...
\end{proof}

\begin{proposition}[Ley de Cancelación]
  Dado los números naturales $a, b, c$ tal que $a + b = a + c$. Entonces se tiene $b = c$.
\end{proposition}

\begin{proof}
  ...
\end{proof}




\subsubsection{Adición y números positivos}

\begin{definition}
  Un número natural $n$ es positivo si y solo si (sii) no es igual a $0$.
\end{definition}

\begin{proposition}\label{prop:ad.aposbnat}
  Si $a$ es un número positivo y $b$ es un número natural, entonces $a + b$ es positivo ($b + a$ también lo es, por la Proposición \ref{prop:ad.conmut}).
\end{proposition}

\begin{proof}
  ...
\end{proof}

\begin{corollary}
  Si $a$ y $b$ son números naturales tal que $a + b = 0$, entonces $a = 0$ y $b = 0$.
\end{corollary}

\begin{proof}
  Suponiendo por contradicción que $a \neq 0$ o $b \neq 0$. Si $a$ es positivo, $a + b$ es positivo por la Proposición \ref{prop:ad.aposbnat}, una contradicción. De manera similar, si $b$ es positivo, $a + b$ es positivo, una contradicción. Así, ambos, $a$ y $b$, deben ser $0$.
\end{proof}

\begin{definition}
  Dado los números naturales $n$ y $m$. $n$ es mayor que o igual a $m$, $n \geq m$ o $m \leq n$, sii $n = m + a$ para algún número natural $a$. Se dice que $n$ es estrictamente mayor que $m$, $n > m$ o $m < n$, sii $n \geq m$ y $n \neq m$.
\end{definition}

\begin{proposition}[Propiedades básicas del orden]
  Dado $a, b, c$ números naturales.

  \begin{enumerate}
  \item Entonces $a \geq a$.
  \item También, si $a \geq b$ y $b \geq c$, entonces $a \geq c$.
  \item Si $a \geq b$ y $b \geq a$, entonces $a = b$.
  \item Se tiene $a \geq b$ si y solo si $a + c \geq b + c$.
  \item Y, se tiene $a < b$ si y solo si $a\!+\!\!+ \leq b$.
  \end{enumerate}
\end{proposition}

\begin{proof}
  ...
\end{proof}

Se puede usar la Proposición 8 para demostrar que $n > m$ si y solo si $n = m + a$ para algún número positivo $a$.

\begin{proposition}[Tricotomía del orden]
  Dado los número naturales $a$ y $b$. Entonces sola una de las afirmaciones es cierta: $a < b$, $a = b$ o $a > b$.
\end{proposition}

\begin{proof}
  ...
\end{proof}




\subsection{Multiplicación}

\begin{definition}
  $m$ es un número natural. Para multiplicar cero por $m$, se define $0 \times m := 0$. Suponiendo inductivamente que se tiene definido multiplicar $n$ a $m$. Entonces, para multiplcar $n\!+\!\!+$ a $m$, se define $(n\!+\!\!+) \times m := (n \times m) + m$.
\end{definition}

\begin{lemma}
  $n \times 0 = 0$.
\end{lemma}

\begin{proof}
  ...
\end{proof}

\begin{lemma}
  $n \times (m\!+\!\!+) = (n \times m) + n$.
\end{lemma}

\begin{proof}
  ...
\end{proof}

\begin{proposition}
  $n \times m = m \times n$.
\end{proposition}

\begin{proof}
  ...
\end{proof}

\begin{proposition}[Ley distributiva]
  Para cualesquiera número naturales $a, b , c$ se tiene $a(b + c) = ab + ac$ y $(b + c)a = ba+ ca$.
\end{proposition}

\begin{proof}
  ...
\end{proof}

\begin{proposition}
  Si $a$ y $b$ son números naturales tal que $a < b$, y $c$ es positivo, entonces $ac < bc$.
\end{proposition}

\begin{proof}
  ...
\end{proof}

\begin{corollary}[Ley de Cancelación]
  Sean $a, b, c$ números naturales tal que $ac = bc$ y $c$ no es cero. Entonces, $a = b$.
\end{corollary}

\begin{proof}
  ...
\end{proof}

\begin{proposition}[Algoritmo euclídeo]
  Dado el número natural $n$ y dado el número positivo $q$. Entonces, existen los números naturales $m$ y $r$ tal que $0 \leq r < q$ y $n = mq + r$.
\end{proposition}


\end{document}
